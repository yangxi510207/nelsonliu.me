\documentclass[12pt, oneside]{article} 
\usepackage{geometry}
\usepackage{hyperref}
\hypersetup{
    colorlinks=true,
    urlcolor=blue
}
 
\urlstyle{same}

\geometry{letterpaper, margin=0.5in} 
\title{Google Summer of Code 2016
  Project Proposal: Addition of various enhancements to the
  \texttt{tree} module by completing stalled pull requests.}
\author{Nelson Liu \\
  University of Washington \\
  nfliu@uw.edu }
\begin{document}
\maketitle
\section{Introduction}
Scikit-learn (as of this writing, 3/17/2015) currently has 435 opened
pull requests and 699 issues. This project is an attempt to take some
of the stalled work in the review pipeline in the theme of the
\texttt{tree} module and complete them to merging quality.\\
To find pull requests to work on, I ran a GitHub search with the
keywords \texttt{'tree' is:pr is:open}. My search filters were
extremely loose and they literally captured any reference of tree;
this resulted in a lot of code regarding nearest neighbors / the
source tree in general to pop up, but I figured that its better to
cast a wider net to ensure that I don't miss any potential work to be
done. The pull requests fetched are listed below in order from oldest
to newest. I used this method of ordering because I figured that older
pull requests are also more likely to be stale, which would make them
good candidates for a Google Summer of Code project. A list of all of
the pull requests, as well as my commentary on them, can be found in
this document: \textbf{\href{http://nelsonliu.me/files/tree-pr-notes.pdf}{A survey of scikit-learn tree-related
    pull requests}}.


\section{Project Goals}
This project seeks to improve the scikit-learn codebase specifically
through the addressing of the outstanding \texttt{tree} module-related
pull requests. The potential benefits of this project are many; the
\texttt{tree} module is improved by merging several stalled pull
requests, and it would cut down on the amount of outstanding pull
requests and issues.

\subsection{List of Deliverables}
By the end of this project, the following will have been completed:
\begin{enumerate}
  \item 
  Finish up and merge stale pull request 
  \textbf{\href{https://github.com/scikit-learn/scikit-learn/pull/941}
    {\#941 Adding a pruning method to the tree}}.
  \begin{enumerate}
    \item Implement decision tree pruning algorithm
    \item Write tests
    \item Write documentation explaining what it is and how to use it.
    \item Create an example illustrating its use.
  \end{enumerate}
  \item 
  Finish up and merge stale pull request
  \textbf{\href{https://github.com/scikit-learn/scikit-learn/pull/6039}
    {\#6039 Add mae regression tree}}
  \begin{enumerate}
    \item Implement MAE regression tree
    \item Write tests
    \item Write documentation explaining what it is and how to use it.
  \end{enumerate}
  \item
  Finish up and merge stale pull request
  \textbf{\href{https://github.com/scikit-learn/scikit-learn/pull/6169}
    {\#6169 BestFirstTreeBuilder should ignore tree.max\_depth}}
  \item
  Finish up and merge stale pull request
  \textbf{\href{https://github.com/scikit-learn/scikit-learn/pull/5181}
    {\#5181 Balanced Random Forest}}
  \begin{enumerate}
    \item Implement Balanced Random Forest
    \item Write tests
    \item Write documentation explaining what it is and how to use it.
    \item Create an example illustrating its use, specifically
    comparing performance vs a regular tree on unbalanced data.
  \end{enumerate}
  \item Participate in project-wide code review/bug-fixes/discussion.
\end{enumerate}

\section{Implementation Plans}
\begin{enumerate}
  \item
  \textbf{\href{https://github.com/scikit-learn/scikit-learn/pull/941}
    {\#941 Adding a pruning method to the tree}}.\\
  \textbf{Motivation:}\\
  Decision Trees are often used for their ease of
  interpretation. Pruning methods are especially useful in this regard
  as they aid both interpretability and generalization performance.\\
  The demand for pruning methods is clearly there; a pull request was
  first opened in July 2012, but stalled without being merged into the
  codebase. In February of 2016, two scikit-learn users commented on
  this original pull request, expressing their desire for decision
  tree pruning to be added to the codebase. Additionally, @amueller
  remarks that he "frequently get[s] asked about post-pruning". With
  this combination of potential for practical application and high
  community demand, I think that it would be a great idea to dust off
  \textbf{\href{https://github.com/scikit-learn/scikit-learn/pull/941}
    {\#941}} and implement post-pruning for decision trees.\\
  \textbf{Implementation Plan:}\\
  Unfortunately,
  \textbf{\href{https://github.com/scikit-learn/scikit-learn/pull/941}
    {\#941}} was written with regards to the old decision tree
  implementation; everything has since been ported to Cython. As a
  result, it will not be possible to reuse most of the code, although
  the ideas within its implementation may prove useful to study.\\
  I plan on reviewing the literature regarding decision tree pruning
  to ascertain the best practices, and then replicate the algorithm in
  the context of scikit-learn. After this is completed, I will write
  unit tests to ensure proper functioning of the method. Lastly, I
  will create add details about the method to the docs, and devise an
  example clearly illustrating how pruning works and its effects.
  
  \item
  \textbf{\href{https://github.com/scikit-learn/scikit-learn/pull/6039}
    {\#6039 Add mae regression tree}}\\
  \textbf{Motivation}:\\
  Currently, the DecisionTreeRegression estimator only supports mean
  squared error (MSE). There have been several requests for new
  criterion, such as mean absolute error (MAE) and mean square of
  percentage error (MSPE) (for example, see
  \textbf{\href{https://github.com/scikit-learn/scikit-learn/issues/5368}
    {\#5368 Request more criterion for random forest
      regression}}). Adding new criterion would enable their use in
  both the decision tree classes and tree based estimators such as
  \texttt{RandomForest}. Additionally, various criterion are better
  suited for different applications.\\
  \textbf{Implementation Plan}:\\
  \textbf{\href{https://github.com/scikit-learn/scikit-learn/pull/6039}
    {\#6039}} has some work already done, but it seems like a fair
  amount is still missing from the patch. Reusing code from the
  existing PR may be the easiest option, but I am not afraid to
  rewrite the code / reanalyze the best way to implement this feature
  if necessary. After adding the new criterion, I will write
  appropriate unit tests and add clarifying information to the docs
  about the new criterion, how they work, and what their applications
  are.

  \item
  \textbf{\href{https://github.com/scikit-learn/scikit-learn/pull/6169}
    {\#6169 BestFirstTreeBuilder should ignore tree.max\_depth}}\\
  \textbf{Motivation}:\\
  It is not quite clear in the docs how
  \texttt{tree.max\_depth} and \texttt{tree.max\_leaf\_nodes}
  interact, which leads to user confusion. I want to clarify the docs
  in this area regarding how the trees are built, or make code changes
  if necessary.\\
  \textbf{Implementation Plan}:\\
  This should probably be a trivial fix, as developer opinion seems to
  be leaning toward just changing the documentation to avoid creating
  a situation where users need to change their code.\\
  If the choice is made to not make any changes in the codebase, I
  will simply update the appropriate docstrings
  (\texttt{BestFirstTreebuilder}, \texttt{DecisionTreeClassifier}, and
  \texttt{DecisionTreeRegressor}).
  If the choice is made to change the codebase to match the docstring
  as it currently is, then I will update the code, write some specific
  tests to ensure that these edge cases work as intended, and perhaps
  devise and example demonstrating the proper behavior.

  \item
  \textbf{\href{https://github.com/scikit-learn/scikit-learn/pull/5181}
    {\#5181 Balanced Random Forest}}\\
  \textbf{Motivation}:\\
  This PR is tangentially related to the tree code, but I believe it
  is quite a useful feature and should be implemented into the main
  codebase. Balanced Random Forests are useful for learning unbalanced
  data, and is used by simply passing a parameter
  \texttt{balanced=True} to \texttt{RandomForestClassifier}.\\
  \textbf{Implementation Plan}:\\
  The original pull request
  \textbf{\href{https://github.com/scikit-learn/scikit-learn/pull/5181}
    {\#5181 Balanced Random Forest}} has a fair amount of work done,
  but I will rewrite portions of code if necessary to optimize for
  efficiency and clarity. Beyond that, I will be writing unit tests to
  verify proper function, documentation explaining the purpose of
  Balanced Random Forests, and an example illustrating its benefits
  over a normal Random Forest on imbalanced data.

  \item \textbf{Participate in project-wide code
    review/bug-fixes/discussion}\\
  I am planning to implement at the least one (possibly trivial) pull
  request each week to bring about improvements in the code-base
  incrementally. I also plan on continuing my participation in
  discussion regarding issues and pull requests raised, and looking
  over the code of other contributors.

\end{enumerate}
\section{Timeline}
During each of these weeks, I will be submitting at least one pull
request unrelated to the \texttt{tree} module / my Google Summer of Code project. These
weeks are not hard deadlines; if I finish a deadline early, I will
move on to the next. In other words, this timeline is quite flexible
and I will adjust it if necessary depending on my summer progress.
\begin{itemize}
  \item 
  \textbf{Community Bonding Period (April 22, 2016 - May 22,
    2016)}\\
  During this period, I will continue contributing to
  scikit-learn by submitting high quality pull requests and issues,
  participating in associated discussion, and reviewing pull requests
  submitted by other contributors. Additionally, I plan to start
  investigation into various aspects of my Google Summer of Code
  project, such as literature reviews about tree pruning algorithms
  and criterion.
  \item
  \textbf{Week 1-3 (May 23, 2016 - June 12, 2016)}\\
  During these weeks, I plan on working on the tree pruning
  algorithm. I want to ensure that it is a high quality addition to
  the API, and will thus spend three weeks on performing the
  literature review and implementing the efficient, clear, and clean
  code as well as verifying its correctness.
  \item
  \textbf{Week 4-7 (May June 13, 2016 - July 10, 2016)}\\
  During these weeks, I plan to add details about the tree
  pruning algorithm to the documentation, write a separate entry in
  the user guide about how the algorithm works, when to use it, and
  what it does. I will also be creating an example illustrating its
  benefits, and write appropriate unit tests to verify its functional
  correctness. Lastly, I will use this time to iron out any last kinks
  in the tree pruning code. I hope to have the tree pruning code and
  docs merged at the end of week 7.
  \item
  \textbf{Week 8-9 (July 11, 2016 - July 24, 2016)}\\
  During these weeks, I plan to work on the MAE regression tree. I
  plan to review literature regarding MAE and study
  implementations online to draw the best elements from these eclectic
  sources for use in a scikit-learn MAE regression tree. I plan on
  spending thse weeks and the next on implementing the tree code and
  verifying its correctness.
  \item \textbf{Week 10-11 (July 25, 2016 - August 7, 2016)}\\
  During this week, I will write unit tests to verify edge case
  coverage and add details about the criterion to the documentation. I
  will also use this time to iron out any last issues with the code,
  and I plan to have the MAE regression tree merged at the end of Week
  11.
  \item \textbf{Week 12 (August 8, 2016 - August 14, 2016)}\\
  During this week, I want to address
  \textbf{\href{https://github.com/scikit-learn/scikit-learn/pull/6169}
    {\#6169 BestFirstTreeBuilder should ignore tree.max\_depth}} and
  get it merged into master.
  \item
  \textbf{Week 13-14 (August 15, 2016 - August 23, 2016)}\\
  During this week, I plan on finishing up pull request
  \textbf{\href{https://github.com/scikit-learn/scikit-learn/pull/5181}
    {\#5181 Balanced Random Forest}} and merging the balanced Random
  Forest parameter into the codebase.

\end{itemize}
\section{Student Information}
\subsection{Personal Information}
Name: Nelson Liu\\
Email: \href{mailto:nfliu@uw.edu}{nfliu@uw.edu}\\
Telephone: +1 714 306 9521\\
Time Zone: GMT/UTC -08:00 (PST, I'm usually in Seattle, San Francisco,
or Los Angeles)\\
IRC Nick: nelson-liu on \href{https://webchat.freenode.net/}{freenode}\\
Github handle: \href{https://github.com/nelson-liu}{nelson-liu}\\
Blog (Currently empty, I have a few unpublished drafts of posts to finalize):
\href{http://blog.nelsonliu.me}{https://blog.nelsonliu.me}\\
\subsection{University Information}
Name: \href{http://www.washington.edu/}{University of Washington, Seattle}\\
Degree: Bachelor of Science / Bachelor of Arts\\
Major: Computer Science, Statistics, Linguistics
\subsection{About Me}
My name is \href{http://nelsonliu.me/}{Nelson Liu}, and I'm a current
undergraduate at the \href{http://www.washington.edu/}{University of
  Washington, Seattle}, studying computer science,
statistics, and linguistics. \\
I have done two summers of NLP research on automated natural language
question-answering with
\href{http://groups.csail.mit.edu/infolab/}{MIT CSAIL's InfoLab
  Group}. I'm currently a member of
\href{http://www.ark.cs.washington.edu/}{Noah Smith's ARK research
  group}, where I work on natural language processing for social
science. My current project involve building a model for prediction of
Supreme Court case decisions based
on submitted amicus briefs.\\
Beyond my participation in research, I'm an avid hacker -- I love
building personal side projects in my free time and participating in
collegiate hackathons. For example, I was Top 20 / 200+ at
HackPrinceton 2015, Top 30 / 300+ at PennApps Winter 2015, and a
finalist at AngelHack Seattle 2015. I work quickly and efficiently,
and love to learn new things.\\
I started contributing to scikit-learn in November of 2015, but I've
been working with the library since early 2014. Scikit-learn is an
awesome community to work with, and I believe that its mission of
democratizing machine learning methods is an important one. I'd love
to contribute my skills to the project as a Google Summer
of Code student.\\
I think I'm a great fit for this project because of my experience with
the library, machine learning experience, and familiarity with the
contribution and review process. I'm committed to staying in constant
contact with my project mentor to ensure that we reach optimal
implementations in the allotted time.
\end{document}
